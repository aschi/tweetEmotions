\section{Einleitung}
\subsection{Thema, Motivation und Ziel der Arbeit}
Als Basis für viele soziale Netzwerkdatenanalysen ist es wichtig erkennen zu können welche Gefühle mit bestimmten Textfragmenten verknüpft werden.
Es gibt verschiedene Ansätze und Algorithmen um dies zu erreichen. Im Rahmen dieser Seminararbeit möchte ich mich mit verschiedenen Algorithmen 
und Ansätzen zum Erkennen einer negativen oder positiven Grundhaltung beziehungsweise Grundstimmung in kurzen Textfragmenten festzustellen 
auseinandersetzen.

Ziel der Arbeit ist es zuerst verschiedene Verfahren zu untersuchen und kennenzulernen um danach entscheiden zu können welche Verfahren im 
Rahmen dieser Arbeit implementiert werden könnten. Schlussendlich sollen die umsetzbaren Verfahren implementiert werden.

Die Implementation soll dabei folgende Funktionalität aufweisen:
\begin{enumerate}
\item In einem ersten Schritt wird ein Thema erfasst. Ein Thema beinhaltet einen „Titel“, „Stichwörter“ und einem Zeitrahmen. Nach diesen Attributen sollten Tweets auf Twitter gefunden werden können.
\item Das Programm soll mit diesen Daten relevante Tweets herauslesen.
\item Nun sollen Parallel verschieden Algorithmen angestossen werden welche entscheiden ob die definierten Tweets eher positiv oder eher negativ sind.
\item Dem Benutzer wird ausgegeben welche Algorithmen wieviele Tweets aus der zuvor zusammengestellten Menge „positiv“ oder „negativ“ bewerten.
\end{enumerate}

\subsection{Planung}
\begin{table}[H]
\begin{center}
\begin{tabular}{|l|l|}
	\hline
	Informationen beschaffen \& Einarbeitung in Twitter & \\
	API und Algorithmen & 13.03.2014 – 20.03.2014\\ \hline
	Erstellen eines Konzepts, Festlegen welche Algorithmen & \\
	implementiert werden sollen bzw können. & 20.03.2014 – 02.04.2014\\ \hline
	\textbf{Milestone 1: Konzept erstellt} & \textbf{02.04.2014}\\ \hline
	Analyse und Design der Software & 03.04.2014 – 16.04.2014\\ \hline
	Implementation & 17.04.2014 – 07.05.2014\\ \hline
	Testing / Dokumentation & 08.05.2014 – 14.05.2014\\ \hline
	\textbf{Milestone 2: Implementation abgeschlossen} & \textbf{14.05.2014} \\ \hline
	Vergleich der Ergebnisse & 15.05.2014 – 21.05.2014\\ \hline
	Dokumentation & 22.05.2014 – 28.05.2014\\ \hline
	Endkorrektur / Abschluss & 29.05.2014 – 16.06.2014\\ \hline
	\textbf{Milestone 3: Arbeit fertig} & \textbf{16.06.2014} \\ \hline
	Abgabe der Arbeit & 16.06.2014 \\ \hline
	Präsentation & 19.06.2014 \\ \hline
\end{tabular}
\caption{Projektplanung}
\end{center}
\end{table}
