\clearpage
\section{Fazit}
Trotz anfänglichen Schwierigkeiten (Twitter API bietet keine Zeitraumsuche, keine Antwort vom SentiStrength Team) ist es gelungen einige Algorithmen zur Erkennung von Emotionen in Texten einzubinden. Es konnten sowohl Unterschiede zwischen den Resultaten der verschiedenen Algorithmen als auch die Performance Vorteile der Parallelisierung mit MPI aufgezeigt werden. Die Bewertung der Unterschiede der verschiedenen Algorithmen oder die Bewertung der Eignung der Algorithmen zum Erkennen von Emotionen in Tweets, würden einen Umfangreichen Datenkorpus mit gelabelten Tweets zu verschiedensten Themengebieten erfordern, war nicht Teil dieser Arbeit und könnte in weiteren Studien untersucht werden.

Durch äussere Umstände konnte das Projekt nicht komplett entsprechend der Anfänglichen Planung (Siehe Kapitel \ref{subsec:planung}) abgewickelt werden. Viele Termine wurden verschoben was dazu geführt hat, dass weniger Zeit im \flqq Reserveblock\frqq Endkorrektur und Abschluss zu Verfügung stand. Trotz diesen kleinen Terminverschiebungen, konnte sowohl das Produkt zum Analysieren von Twittermeldungen als auch die hier vorliegende Dokumentation im Zeitrahmen abgeschlossen werden. 