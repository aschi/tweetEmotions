\clearpage
\section{Grundlagen}
\subsection{Algorithmen zur Erkennung der Gefühlslage in Texten}
Es wird grundsätzlich zwischen zwei Ansätzen zur Analyse der Gefühlslage in Texten unterschieden. Der erste Ansatz verwendet machine-learning Technologien der zweite basiert auf der lexikalischen Analyse der Texte. Für den ersten Ansatz wird ein Korpus mit gelabelten Trainingsdaten benötigt. \cite{thumbs_up} Der Hauptvorteil von machine-learning basierten Methoden ist, dass sie sich an neue Gegebeneheiten anpassen können. Der grosse Nachteil ist, dass keine gelabelten Trainingsdaten für neue Themen existieren und dass das erstellen eines neuen Datenkorpus sehr aufwändig und teuer ist. Die lexikalischen Methoden haben eine vordefinierte Liste von Wörtern welche mit bestimmten Gefühlen verknüpft sind. Um gute Resultate zu erhalten, müssen diese Wortlisten der entsprechenden Domäne angepasst werden. So werden zum Beispiel im Web-Slang andere Wörter mit anderen Gefühlen verknüpft als in einem formalen Text. \cite{comparing}

Im Rahmen dieser Arbeit werden sowohl machine-learning basierte als auch lexikalische Algorithmen zur Erkennung der Gefühlslage in Texten untersucht und - wenn in Python umsetzbar - auf Tweets zu verschiedenen Themen angewandt.

\subsubsection{Emoticons}
Der einfachste Ansatz zum erkennen der Grundhaltung des Authors gegenüber eines Themas ist die Untersuchung der Emoticons die es beinhaltet. Emoticons sind meist Kombinationen von ASCII Zeichen die einen Gesichtsausdruck darstellen. Dieser kann Gefühle unter anderem gefühle wie glücklich oder traurig representieren. Für diesen Algorithmus wurde eine Menge von gängigen Emoticons aus dem Web \cite{emoticons1}\cite{emoticons2}\cite{emoticons3}\cite{emoticons4} zusammengesucht und manuell in \flqq positiv\frqq, \flqq neutral\frqq, und \flqq negativ\frqq klassifiziert. Mithilfe solcher Listen kann die Anzahl Emoticons der entsprechenden Kategorie in einem Text zählen und gegeneinander Abwägen. \cite{comparing} Die Umsetzung eines solchen Algorithmus in Python stellt keine Probleme dar.

\subsubsection{LIWC - Linguistic Inquiry and Word Count}
LIWC ist ein Textanalysetool welche emotionale, kognitive und strukturelle Komponenten von Texten mithilfe eines Wörterbuchs klassifiziert. Die LIWC Software wird kommerziell vertrieben und kann - bei Bedarf - um eigene Wörterbücher ergänzt werden. LIWC liest Text ein und generiert daraus ein Tab-Delimited File welches zum Beispiel in einem Python Programm oder mithilfe von SPSS weiter verarbeitet werden könnte. Es existieren Versionen für Windows und Macintosh Computer. \cite{comparing}\cite{liwcfaq} Die fehlende Verfügbarkeit für Linux, die komplizierte Anbindung an ein Python Programm und ökonomische Gründe haben dazu geführt, dass im Rahmen dieser Arbeit keine LIWC-Python Anbindung implementiert wurde.

\subsubsection{SentiStrength}
SentiStrength ist ein machine-learning basiertes Textanalyse Tool welches spezialisiert ist auf die Analyse von Texten aus dem Social-Web. Die Freie Version ist auf der Basis des .Net Technologiestacks implementiert und nur für Windows verfügbar. Eine kommerzielle Version des Tools wurde in Java geschrieben. Die kommerzielle Java Version ist für Forschungszwecke frei verfügbar und kann per Email angefordert werden. Diese Version liesse sich wie im Listing \ref{lst:python_sentistrength} aufgezeigt an Python Programme anbinden. \cite{sentistrength}\cite{comparing}

\begin{lstlisting}[language=Python, caption={Python Anbindung an SentiStrength (JAVA)}, label={lst:python_sentistrength}]
#Alec Larsen - University of the Witwatersrand, South Africa, 2012 import shlex, subprocess

def RateSentiment(sentiString):
    #open a subprocess using shlex to get the command line string into the correct args list format
    p = subprocess.Popen(shlex.split("java -jar SentiStrength.jar stdin sentidata C:/SentStrength_Data/"),stdin=subprocess.PIPE,stdout=subprocess.PIPE,stderr=subprocess.PIPE)
    #communicate via stdin the string to be rated. Note that all spaces are replaced with +
    stdout_text, stderr_text = p.communicate(sentiString.replace(" ","+"))
    #remove the tab spacing between the positive and negative ratings. e.g. 1-5 -> 1-5
    stdout_text = stdout_text.rstrip().replace("\t","")
    return stdout_text
\end{lstlisting}

Der entsprechende Email-Kontakt hat nicht innert nützlicher Frist geantwortet, weshalb dieser Algorithmus aus dieser Arbeit ausgeklammert werden musste.

\subsubsection{SentiWordNet}
Das SentiWordNet ist eine offen verfügbare lexikalische Ressource (Wörterbuch) auf der Basis von WordNet\cite{wordnet} mithilfe welchem ein Text auf folgendes untersucht werden kann:
\begin{itemize}
\item Subjektivität-Objektivität: Besteht ein gegebener Text primär aus Fakten, oder haften ihm Emotionen an?
\item Positiv-Negativ: Sind die Emotionen die im Text ausgedrückt werden primär positiv oder primär negativ?
\item Stärke der Emotion: Wie stark ist die positive oder negative Emotion in einem Text?
\end{itemize}   
Das Wörterbuch wird als Tabulator-getrenntes Textfile unter \url{http://sentiwordnet.isti.cnr.it/download.php} zu Verfügung gestellt. 
Das Wörterbuch enthält folgende Spalten:
\begin{itemize}
\item die Wortart ('a' = Adjektiv, 'n' = Nomen, ...)
\item eine ID (z.B. 00004980)
\item einen Wert der die Stärke der positiven Emotionen anzeigt (Zwischen 0 und 1, folgend auch $posScore$ genannt)
\item einen Wert der die Stärke der negativen Emotionen anzeigt (Zwischen 0 und 1, folgend auch $negScore$ genannt)
\item das Synset (Wörter mit der selben Bedeutung und jeweils einem \flqq Rank\frqq)
\item eine Liste von Synonymen
\item eine Beschreibung des Wortes
\end{itemize}

Den Wert für die Objektivität ($objScore$) lässt sich wie folgt berechnen: 
\begin{equation}
	objScore = 1-(negScore+posScore)
\end{equation}

Zum besseren Verständnis sind im Listing \ref{lst:sentiwordnetlines} drei Zeilen aus dem aktuellen SentiWordNet Wörterbuch abgebildet.

\begin{lstlisting}[language=Java, showtabs=true, caption={SentiWordNet Zeile)}, label={lst:sentiwordnetlines}]
a	00004980	0	0	unabridged#1	(used of texts) not shortened; "an unabridged novel"
a	00005107	0.5	0	uncut#7 full-length#2	complete; "the full-length play"
a	00007813	0	0.5	nonabsorptive#1 nonabsorbent#1	not capable of absorbing or soaking up (liquids)
\end{lstlisting}

Die Gute Dokumentation und die freie Verfügbarkeit aller Ressourcen lassen eine Python Implementation des SentiWordNet Wörterbuches zu.

\subsubsection{SenticNet}
SenticNet ist eine weitere lexikalische Resource welche im Opinion Mining verwendet werden kann. Das SenticNet kann unter \url{http://sentic.net/downloads/} heruntergeladen werden. Das Wörterbuch ist in diesem Falle als XML abgespeichert. Ein Wort-Eintrag enthält neben dem Wort selbst vor allem vier Werte für pleasentness, attention, sensitivity und aptitude. Aus diesen Werten lässt sich mit folgender Formel die allgemeine Polarität (positiv oder negativ) berechnen:\cite{senticnet}

\begin{equation}
	p = \sum_{i=1}^{N} \frac{Plsnt(c_i)+|Attnt(c_i)|-|Snst(c_i)|+Aptit(c_i)}{3N}
\end{equation}

In Version 2 des SenticNet wird dieser Wert aus praktischen Gründen jeweils pro Wort direkt im Wörterbuch mitgeliefert. Um die Polarität eines ganzen Textes zu berechnen wird jedoch weiterhin die oben genannte Formel empfohlen. \cite{senticnet} Zum einfacheren Verständnis befindet sich unter Listing \ref{lst:senticnetword} Beispielhaft ein Eintrag eines Wortes.  

\begin{lstlisting}[language=XML, caption={SenticNet Wort}, label={lst:senticnetword}]
<rdf:Description rdf:about="http://sentic.net/api/en/concept/worship">
	<rdf:type rdf:resource="http://sentic.net/api/concept"/>
	<text xmlns="http://sentic.net/api/">worship</text>
	<semantics xmlns="http://sentic.net/api/" rdf:resource="http://sentic.net/api/en/concept/hope"/>
	<semantics xmlns="http://sentic.net/api/" rdf:resource="http://sentic.net/api/en/concept/religious_purpose"/>
	<semantics xmlns="http://sentic.net/api/" rdf:resource="http://sentic.net/api/en/concept/trust"/>
	<semantics xmlns="http://sentic.net/api/" rdf:resource="http://sentic.net/api/en/concept/devotion"/>
	<semantics xmlns="http://sentic.net/api/" rdf:resource="http://sentic.net/api/en/concept/religious"/>
	<pleasantness xmlns="http://sentic.net/api/" rdf:datatype="http://www.w3.org/2001/XMLSchema#float">+0.265</pleasantness>
	<attention xmlns="http://sentic.net/api/" rdf:datatype="http://www.w3.org/2001/XMLSchema#float">+0.601</attention>
	<sensitivity xmlns="http://sentic.net/api/" rdf:datatype="http://www.w3.org/2001/XMLSchema#float">-0.207</sensitivity>
	<aptitude xmlns="http://sentic.net/api/" rdf:datatype="http://www.w3.org/2001/XMLSchema#float">+0.373</aptitude>
	<polarity xmlns="http://sentic.net/api/" rdf:datatype="http://www.w3.org/2001/XMLSchema#float">+0.344</polarity>
</rdf:Description>
\end{lstlisting}

Alle Grundlagen und das Wörterbuch des SenticNet sind frei zugänglich, weshalb sich auch hier eine entsprechende Python Implementation anbot.

\subsubsection{SASA - SailAil Sentiment Analyzer}
Der SailAil Sentiment Analyzer ist eine machine-learning basierte Library die direkt für Python verfügbar ist. In dieser Arbeit wurde die Library in der Version 0.1.3 verwendet (\url{https://pypi.python.org/pypi/sasa/0.1.3})

\subsubsection{Klassifizierung mit NLTK und naivem Bayes}
\label{subsubsec:grundlagennaivebayes}
Bei der Recherche zu dieser Arbeit bin ich über einen Artikel von Jacob Perkins \cite{nltkbayes} gestolpert. Er beschreibt wie man mithilfe des Natural Language Toolkits (NLTK) \cite{nltk} einen einfachen Sentiment Analyzer implementieren kann. Er trainiert einen \lstinline$nltk.classify.NaiveBayesClassifier$ mithilfe des NLTK eigenen Film-Review Korpus (\lstinline$nltk.corpus.movie_reviews$).

\subsection{Twitter Search API}
\label{subsec:grundlagentwitter}
Das Twitter Search API\cite{twitterapi} ist ein REST API welches es erlaubt Tweets zu bestimmten Themen zu suchen. Christian Koepp hat eine Python Anbindung an dieses API implementiert\cite{twittersearch}. Mithilfe dieser Anbindung lässt sich in wenigen Zeilen (Siehe Listing \ref{lst:twitterSearch}) eine Twitter Anbindung realisieren.

\begin{lstlisting}[language=Python, caption={TwitterSearch Python-Twitter Anbindung}, label={lst:twitterSearch}]
from TwitterSearch import *
try:
    tso = TwitterSearchOrder()
    tso.setKeywords(['Worldcup', 'Brazil'])
    tso.setLanguage('en')
    tso.setIncludeEntities(False)

    ts = TwitterSearch(
    	consumer_key = 'aaabbb', 
    	consumer_secret = 'cccddd', 
    	access_token = '111222', 
    	access_token_secret = '333444'
    )

    for tweet in ts.searchTweetsIterable(tso):
        print( '@%s tweeted: %s' % ( tweet['user']['screen_name'], tweet['text'] ) )
except TwitterSearchException as e:
    print(e)
\end{lstlisting}

Was das Twitter Search API nicht bietet, ist die Möglichkeit nach Tweets in bestimmten Zeiträumen zu suchen. Man sucht jeweils im aktuellen \lstinline$SearchIndex$ welcher sowohl aus populären als auch sehr aktuellen Tweets besteht. Für die in der Aufgabenstellung beschriebene Zeitraum-Auswahl-Funktion könnten Daten von einem kommerziellen Anbieter bezogen werden oder man könnte sich selber ein Twitter Archiv mithilfe der Streaming API \cite{twitterstreaming} anlegen. Im Rahmen dieser Arbeit wurde darauf verzichtet. Das heisst, in der im Rahmen dieser Arbeit erstellten Applikation kann \flqq nur\frqq nach aktuellen oder populären Tweets mithilfe von Stichwörtern gesucht werden. 